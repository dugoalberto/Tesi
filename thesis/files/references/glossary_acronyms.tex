% Acronyms
\newacronym[description={\glslink{llmg}{Large Language Model}}]{llm}{LLM}{Large Language Model}
\newacronym[description={\glslink{ideg}{Integrated Development Environment}}]{ide}{IDE}{Integrated Development Environment}
\newacronym[description={\glslink{LoRAg}{Low-Rank Adaptation}}]{lora}{LoRA}{Low-Rank Adaptation}

% Glossary
\newglossaryentry{llmg}{
    name=\glslink{llm}{LLM},
    text={Large Language Model},
    sort=llm,
    description={Un Large Language Model (LLM) è un modello capace di generare testi in linguaggio naturale basandosi su modelli statistici.
    Questi modelli acquisiscono una conoscenza linguistica attraverso l'apprendimento di relazioni statistiche durante un processo di addestramento computazionalmente oneroso.}
}

\newglossaryentry{ideg}{
    name=\glslink{ide}{IDE},
    text={Integrated Development Environment},
    sort=ide,
    description={Un Integrated Development Environment (IDE) è un'applicazione software che fornisce servizi per facilitare lo sviluppo di software. Un IDE generalmente comprende un editor di codice sorgente, strumenti di compilazione e debugging e un ambiente per eseguire il software in sviluppo.}
}


\newglossaryentry{LoRAg}{
    name=\glslink{lora}{LoRA},,
    text={Low-Rank Adaptation},
    sort=LoRA,
    description={ADD DESCRIPTION.}
}


\newglossaryentry{fine-tuning}{
    name={Fine-tuning},
    text={fine-tuning},
    sort=fine-tuning,
    description={ADD DESCRIPTION.}
}



\newglossaryentry{mclearning}{
    name={Machine learning},
    text={machine learning},
    sort=Machine learning,
    description={branca dell'\textit{intelligenza artificiale} che utilizza metodi statistici per migliorare la performance di un algoritmo nell'identificare pattern nei dati, imparando da questi a svolgere delle funzioni piuttosto che attraverso la programmazione esplicita.}
}



\newglossaryentry{prototipog}{
    name={Prototipo},
    text={prototipo},
    sort=prototipo,
    description={Un prototipo è un esemplare o un modello di un prodotto o di un sistema che viene realizzato antecedentemente al prodotto finale}
}

