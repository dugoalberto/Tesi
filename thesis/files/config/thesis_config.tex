% Load variables
\newcommand{\myUni}{Università degli Studi di Padova}
\newcommand{\myDepartment}{Dipartimento di Matematica ``Tullio Levi-Civita''}
\newcommand{\myFaculty}{Corso di Laurea in Informatica}
\newcommand{\myTitle}{Test Automatici con Large Language Model}
\newcommand{\myDegree}{Tesi di Laurea Triennale}
\newcommand{\profTitle}{Prof.}
\newcommand{\myProf}{Ballan Lamberto}
\newcommand{\graduateTitle}{Laureando}
\newcommand{\myName}{Dugo Alberto}
\newcommand{\myStudentID}{2042382}
\newcommand{\myAA}{2023-2024}
\newcommand{\myLocation}{Padova}
\newcommand{\myTime}{Luglio 2024}
% Acronyms
\newacronym[description={\glslink{llmg}{Large Language Model}}]{llm}{LLM}{Large Language Model}
\newacronym[description={\glslink{ideg}{Integrated Development Environment}}]{ide}{IDE}{Integrated Development Environment}
\newacronym[description={\glslink{LoRAg}{Low-Rank Adaptation}}]{lora}{LoRA}{Low-Rank Adaptation}
\newacronym[description={\glslink{llmse}{Assured LLM-Based Software Engineering}}]{llmse}{LLMSE}{Assured LLM-Based Software Engineering}
\newacronym[description={\glslink{peft}{Parameter Efficient Fine-Tuning}}]{peft}{PEFT}{Parameter Efficient Fine-Tuning}
% Glossary
\newglossaryentry{llmg}{
    name=\glslink{llm}{LLM},
    text={Large Language Model},
    sort=llm,
    description={Un Large Language Model (LLM) è un modello capace di generare testi in linguaggio naturale basandosi su modelli statistici.
    Questi modelli acquisiscono una conoscenza linguistica attraverso l'apprendimento di relazioni statistiche durante un processo di addestramento computazionalmente oneroso.}
}

\newglossaryentry{ideg}{
    name=\glslink{ide}{IDE},
    text={Integrated Development Environment},
    sort=ide,
    description={Un Integrated Development Environment (IDE) è un'applicazione software che fornisce servizi per facilitare lo sviluppo di software. Un IDE generalmente comprende un editor di codice sorgente, strumenti di compilazione e debugging e un ambiente per eseguire il software in sviluppo.}
}

\newglossaryentry{LoRAg}{
    name=\glslink{lora}{LoRA},,
    text={Low-Rank Adaptation},
    sort=LoRA,
    description={Approccio di \gls{fine-tuning} il quale permette di costruire diversi modelli per \textit{downstream tasks} i quali ne condividono uno pre-addestrato.}
}

\newglossaryentry{fine-tuning}{
    name={Fine-tuning},
    text={fine-tuning},
    sort=fine-tuning,
    description={Metodologia che permette, attraverso modifiche minimali agli iperparametri di una \textit{neural network}, di adattare un modello ad un nuovo dataset senza doverlo riaddestrare, ottenendo quindi un modello più accurato.}
}

\newglossaryentry{mclearning}{
    name={Machine learning},
    text={machine learning},
    sort=Machine learning,
    description={Branca dell'\textit{intelligenza artificiale} che utilizza metodi statistici per migliorare la performance di un algoritmo nell'identificare pattern nei dati, imparando da questi a svolgere delle funzioni piuttosto che attraverso la programmazione esplicita.}
}

\newglossaryentry{prototipog}{
    name={Prototipo},
    text={prototipo},
    sort=prototipo,
    description={Un prototipo è un esemplare o un modello di un prodotto o di un sistema che viene realizzato antecedentemente al prodotto finale}
}

\newglossaryentry{deeplearning}{
    name={Deep learning},
    text={deep learning},
    sort=Deep learning,
    description={ADD DESCRIPTION.}
}

\newglossaryentry{llmseg}{
    name={Assured LLM-Based Software Engineering},
    text={Assured LLM-Based Software Engineering},
    sort=assured LLM-Based Software Engineering,
    description={ADD DESCRIPTION.}
}
\newglossaryentry{peftg}{
    name={Parameter Efficient Fine-Tuning},
    text={Parameter Efficient Fine-Tuning},
    sort=parameter Efficient Fine-Tuning,
    description={ADD DESCRIPTION.}
}

% Define custom colors
\definecolor{hyperColor}{HTML}{930000}
\definecolor{tableGray}{HTML}{F5F5F7}

% Set line height
\linespread{1.5}

% Custom hyphenation rules
\hyphenation {
    e-sem-pio
    ex-am-ple
}

% Images path
\graphicspath{{img/}}

% Force page color, as some editors set a grayish color as default
\pagecolor{white}

% Better spacing for footnotes
\setlength{\skip\footins}{5mm}
\setlength{\footnotesep}{5mm}

\setlength{\headheight}{14.5pt}
\addtolength{\topmargin}{-2.45pt}

% Listings setup
\lstset{
    language=[LaTeX]Tex,%C++,
    keywordstyle=\color{RoyalBlue}, %\bfseries,
    basicstyle=\small\ttfamily,
    %identifierstyle=\color{NavyBlue},
    commentstyle=\color{Green}\ttfamily,
    stringstyle=\rmfamily,
    numbers=none, %left,%
    numberstyle=\scriptsize, %\tiny
    stepnumber=5,
    numbersep=8pt,
    showstringspaces=false,
    breaklines=true,
    frameround=ftff,
    frame=single
}

% Add a subscript G to a glossary entry
\newcommand{\glox}{\textsubscript{\textbf{\textit{G }}}}

% If the subscript G is followed by a punctuation character, or anything else, you need to use \gloxspacing to prevent rendering issues, where the characters collide. Example in Chapter 7
\newcommand{\gloxspacing}{\hspace{-0.3em}}

% Improvements the paragraph command
\titleformat{\paragraph}
{\normalfont\normalsize\bfseries}{\theparagraph}{1em}{}
\titlespacing*{\paragraph}
{0pt}{3.25ex plus 1ex minus .2ex}{1.5ex plus .2ex}

% Define use case environment
\newcounter{usecasecounter} % define a counter
\setcounter{usecasecounter}{0} % set the counter to some initial value
% Parameters
% #1: ID
% #2: Nome
\newenvironment{usecase}[2]{
    \renewcommand{\theusecasecounter}{\usecasename #1}  % this is where the display of the counter is overwritten/modified
    \refstepcounter{usecasecounter} % increment counter
    \vspace{2em}
    \par \noindent % start new paragraph
    {\normalsize \textbf{\usecasename #1: #2}} % display the title before the content of the environment is displayed
    \vspace{.5em}
}{
    \medskip
}
\newcommand{\usecasename}{UC}
\newcommand{\usecaseactors}[1]{\textbf{\\Attori Principali:} #1}
\newcommand{\usecasepre}[1]{\textbf{\\Precondizioni:} #1}
\newcommand{\usecasedesc}[1]{\textbf{\\Descrizione:} #1}
\newcommand{\usecasepost}[1]{\textbf{\\Postcondizioni:} #1}
\newcommand{\usecasealt}[1]{\textbf{\\Scenario Alternativo:} #1}

% Define risks environment
\newcounter{riskcounter} % define a counter
\setcounter{riskcounter}{0} % set the counter to some initial value
% Parameters
% #1: Title
\newenvironment{risk}[1]{
    \refstepcounter{riskcounter} % increment counter
    \par \noindent % start new paragraph
    \textbf{\arabic{riskcounter}. #1} % display the title before the content of the environment is displayed
}{
    \par\medskip
}
\newcommand{\riskname}{Rischio}
\newcommand{\riskdescription}[1]{\textbf{\\Descrizione:} #1.}
\newcommand{\risksolution}[1]{\textbf{\\Soluzione:} #1.}

% Apply fancy styling to pages
\pagestyle{fancy}
\fancyhf{}
\fancyhead[L]{\leftmark} % Places Chapter N. Chatper title on the top left
\fancyfoot[C]{\thepage} % Page number in the center of the footer

% Adds a blank page while increasing the page number
\newcommand\blankpage{ 
    \clearpage
    \begingroup
      \null
      \thispagestyle{empty}
      \hypersetup{pageanchor=false}
      \clearpage
    \endgroup
}

% Increase page numbering
\newcommand\increasepagenumbering{
    \addtocounter{page}{+1}
}

% Make glossaries and bibliography
\makeglossaries
\bibliography{references/bibliography}
\defbibheading{bibliography} {
    \cleardoublepage
    \phantomsection
    \addcontentsline{toc}{chapter}{\bibname}
    \chapter*{\bibname\markboth{\bibname}{\bibname}}
}
\glsaddall

% Set up hyperlinks
\hypersetup{
    colorlinks=true,
    linktocpage=true,
    pdfstartpage=1,
    pdfstartview=,
    breaklinks=true,
    pdfpagemode=UseNone,
    pageanchor=true,
    pdfpagemode=UseOutlines,
    plainpages=false,
    bookmarksnumbered,
    bookmarksopen=true,
    bookmarksopenlevel=1,
    hypertexnames=true,
    pdfhighlight=/O,
    allcolors = hyperColor
}

% Set up captions
\captionsetup{
    tableposition=top,
    figureposition=bottom,
    font=small,
    format=hang,
    labelfont=bf
}