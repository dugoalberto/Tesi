\chapter{Valutazione retrospettiva}
\label{chap:conclusioni}
    \section{Conoscenze acquisite}
    Durante l’esperienza in Zucchetti, sono stati affrontati diversi temi legati all'intelligenza artificiale. I \textit{test} sono stati il tema principale dello stage; infatti, sebbene le tecniche studiate fossero molto diverse tra loro, l'obiettivo principale era migliorare la generazione di \textit{test}.
    Le conoscenze acquisite sono variate notevolmente tra il primo e il secondo macro periodo. Nel primo periodo, sono state apprese nozioni riguardanti gli \gls{llmse}, che costituiscono la base del lavoro sulla generazione di \textit{test}. È stato particolarmente importante delineare un percorso per futuri approfondimenti; comprendere che i \textit{test} generati sono qualitativamente migliori se il codice è in inglese e il linguaggio è tipizzato è stato di fondamentale importanza.
    Nel secondo macro periodo, invece, le nozioni sono state principalmente legate alla specializzazione del modello tramite \gls{fine-tuning}. Sono state apprese tecniche come \gls{lora} e approfondite alcune delle possibili migliorie applicabili, come \textit{MoLE} e \textit{AdaMoLE}.
    Oltre alla specializzazione del modello, è stata esaminata a fondo la quantizzazione. Sono state approfondite le tecniche di quantizzazione asimmetrica e simmetrica e la loro applicazione agli \gls{llm}, al fine di ridurre le loro dimensioni.    
    Le conoscenze acquisite non si sono limitate all'ambito lavorativo, ma hanno arricchito anche il lato personale. Durante lo stage, sono state infatti molte le lezioni apprese, la maggior parte delle quali deriva dall'interazione con i colleghi e il responsabile dello stage, Gregorio Piccoli. Queste esperienze hanno contribuito significativamente alla crescita professionale e personale.
        %cosa ho imparato dallo stage
    \section{Valutazione personale}
            %cosa ho fatto bene -> rispetto obiettivi, integrazione col team
    La valutazione che può essere associata con l'esperienza trascorsa durante questo stage è sicuramente più che positiva.
    Durante il percorso è stato appreso come gestire un progetto in un ambiente di lavoro professionale e innovativo, ricercando efficacia, efficienza e qualità in ciò che doveva essere completato.
    È stato inoltre significativo affacciarsi al mondo della ricerca e sviluppo all'interno di una grande azienda come Zucchetti, poiché rappresenta un segmento molto interessante e dinamico. Questa esperienza ha permesso di comprendere meglio le dinamiche e le sfide legate all'innovazione tecnologica in un contesto aziendale di alto livello.
    Un'ulteriore prova del successo dello stage deriva dal rapporto personale che si è creato con i colleghi. Questo aspetto è di grande rilevanza poiché favorisce il confronto e il miglioramento delle capacità comunicative.
        %cosa avrei potuto fare meglio 
    Non vi sono solo ovviamente solo obiettivi raggiunti e rapporti instaurati, durante un primo incontro con la vita lavorativa all'interno di un'azienda sono emerse anche difficoltà. 
     La principale difficoltà è stata mantenere costanza e dedizione per raggiungere gli obiettivi prefissati, un aspetto spesso sottovalutato.
     Inoltre, sebbene il clima fosse disteso vi è stata la necessità costante di portare risultati di valore con il proprio operato, un aspetto che durante la vita universitaria spesso non si affronta.
    È importante sottolineare che sono state proprio queste piccole difficoltà che hanno accompagnato l'esperienza di stage ad accrescere l'interesse verso lo stage stesso, ponendo ogni giorno un nuovo obiettivo stimolante al fine di migliorare.

\newpage