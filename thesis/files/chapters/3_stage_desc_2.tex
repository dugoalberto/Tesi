\chapter{Fine-tuning di LLM attraverso LoRA e ottimizzazioni}
\label{chap:descrizione-stage-2}
\section{Analisi del dominio applicativo}
    \subsection{Analisi del tema}
    Nella seconda fase del progetto si procederà con il \gls{fine-tuning} di un \gls{llm} attraverso LoRA, e le sue ottimizzazioni, come ad esempio \textit{MoLE} e \textit{AdaMoLE}.
    In questo capitolo si approfondiranno inoltre gli studi effettuati sul \gls{fine-tuning} e quantizzazione, concentrandoci maggiormente sul possibile apporto valoriale che questi ultimi possono dare ad un \gls{llm} e alle sue implementazioni.
    %descrive il dominio del problema da affrontare:
    %la porzione del mondo reale, rilevante per il sistema\\
    %-> Su cui si devono mantenere informazioni\\
    %-> Concuisideve interagire

    \subsection{Esempi di utilizzo}
    \subsection{LoRA}
    \subsubsection{Future applicazioni} 

        %future applicazioni (?)
        %   -> migliorare filtri
        %   -> Genetic Improvement per migliorare prompt e Methauristic algorithm per migliorare candidate solutions
        %   -> prompt engineering
        %   -> in-learning context
\section{Analisi dei requisiti}
    \subsection{Analisi preventiva dei rischi}
    \subsection{Requisiti e obiettivi}


\section{Sviluppo del prodotto}
    \subsection{\textit{Fine-tuning}}
    % come hai fatto e a cosa serve
    % come si effettua fine tuning e quanto è costoso
        \subsubsection{Ottimizzazioni}
        % MoLE
        % AdaMoLE
    \subsection{Quantizzazione}
    % cos'è 
    % a cosa serve
    % cosa si è utilizzato
    \subsection{Documentazione e \textit{test}}
        % documentazione di come si 
        % test del codice generato dal programma stesso <- in modo tale da visionare gli effetti dello script

\section{Resoconto finale}
    \subsection{Prodotti ottenuti}
        %script python per generare test scegliendo le classi e analizzando i vari llm
        %lista prodotto ottenuto
    \subsection{Risultati ottenuti}
        % bene o male, pro e contro del PROGETTO e di come è stato affrontato
    \subsection{Conclusione}
        %conclusione di tutto il lavoro svolto quindi prodotto 
        %problematiche relative al fatto che se non applico filtri di LLMSE il modello non è in grado di generare codice sorgente valido
        %ma non so se questo codice non è valido perchè è errato o perchè ha trovato un bug

\newpage