\chapter{Svolgimento del progetto}
\label{chap:descrizione-stage}
\section{Analisi del dominio applicativo}
    \subsection{Analisi del tema}
    Il tema del progetto riguarda la realizzazione di test automatici per il testing di codice sorgente, ponendosi come
    obiettivo la semplificazione del processo di testing affidato ai programmatori attraverso l'utilizzo di \glspl{llmg}.
    descrive il dominio del problema da affrontare:
    la porzione del mondo reale, rilevante per il sistema\\
    -> Su cui si devono mantenere informazioni\\
    -> Concuisideve interagire
    \subsection{Esempi di utilizzo}
        %Automatizzazione dei test attraverso llm in modo da aumentare il test coverage dei corner case
        %ridurre bug 
        %ridurre tempo di sviluppo dei test 
    \subsection{\textit{Assured LLMSE}}
        %cosa sono 
        %come funzionano 
        %perchè sono importanti per il progetto 
        %distinzione tra off e online 
        %future applicazioni (?)
        %   -> migliorare filtri
        %   -> Genetic Improvement per migliorare prompt e Methauristic algorithm per migliorare candidate solutions
    

\section{Analisi dei requisiti}
    \subsection{Analisi preventiva dei rischi}
    Durante la fase di analisi dei rischi sono stati individuate le possibili criticità che potranno essere riscontrate.
    Si è quindi proceduto a elaborare delle possibili soluzioni per far fronte a tali rischi.

    \begin{risk}{Mancanza di materiale informativo}
        \riskdescription{Trattandosi di una novità nel settore e in fase di crescita, la possibile assenza di materiale informativo relativo all'argomento stesso potrebbe rallentare il processo di apprendimento.}
        \risksolution{coinvolgimento del responsabile a capo del progetto relativo}
        \label{risk:data-absence} 
    \end{risk}

    \subsection{Requisiti e obiettivi}

    \begin{center}
        \rowcolors{1}{}{tableGray}
        \begin{longtable}{|p{2.25cm}|p{7.75cm}|p{2.25cm}|}
        \hline
        \multicolumn{1}{|c|}{\textbf{Obiettivo}} & \multicolumn{1}{c|}{\textbf{Descrizione}}\\ 
        \hline 
        \endfirsthead
        \multicolumn{3}{c}%
        {{\bfseries \tablename\ \thetable{} -- Continuo della tabella}}\\
        \hline
        \multicolumn{1}{|c|}{\textbf{Obiettivo}} & \multicolumn{1}{c|}{Descrizione}\\ \hline 
        \endhead
        \hline
        \multicolumn{3}{|r|}{{Continua nella prossima pagina...}}\\
        \hline
        \endfoot
        \endlastfoot 
        OB 1 & Realizzazione di smoke test in Python generati da codice reale. \\
        \hline
        OB 2 & Realizzazione di decorazioni assert per funzioni. \\
        \hline
        OB 3 & Realizzazione di test a partire da pseudocodice. \\
        \hline
        OB 4 & Test con modello LLM con fine tuning attraverso LoRA. \\
        \hline
        \hiderowcolors
        \caption{Requisiti.}
        \label{tab:requisiti_obbiettivi}
        \end{longtable}
    \end{center}


\section{Sviluppo del prodotto}
    \subsection{\textit{Script}}
    % generazione script per scegliere classi da testare ed eseguire i test
    % generazione di filtri per migliorare i risultati e renderli adatti
    \subsection{\textit{Benchmarks}}
    % benchmarking degli LLM
    \subsection{\textit{Fine-tuning}}
    % come hai fatto e a cosa serve
    \subsection{Quantizzazione}
    % cos'è 
    % a cosa serve
    % cosa si è utilizzato
    \subsection{Documentazione e test automatici}
        % documentazione di come si 
        % test del codice generato dal programma stesso <- in modo tale da visionare gli effetti dello script

\section{Resoconto finale}
    \subsection{Prodotti ottenuti}
        %script python per generare test scegliendo le classi e analizzando i vari llm
        %lista prodotto ottenuto
    \subsection{Risultati ottenuti}
        % bene o male, pro e contro del PROGETTO
    \subsection{Conclusione}
        %conclusione di tutto il lavoro svolto quindi prodotto 
        %problematiche relative al fatto che se non applico filtri di LLMSE il modello non è in grado di generare codice sorgente valido
        %ma non so se questo codice non è valido perchè è errato o perchè ha trovato un bug

\newpage