\chapter{Introduzione}
\label{chap:introduzione}



%Introduzione al contesto applicativo.

%Lorem Figure \ref{fig:entanglement}

%Esempio di utilizzo di un termine nel glossario \gls{api}.

%Esempio di citazione in linea \cite{site:agile-manifesto}.

%Esempio di citazione nel piè di pagina citazione\footcite{womak:lean-thinking}

%\lipsum[1-2]

\section{L'azienda}
\begin{figure}[h!]
    \centering
    \includegraphics[alt={Testo alternativo dell'immagine}, width=0.7\columnwidth]{img/logoZucchetti.jpeg}
    \caption{logo di Zucchetti}
    \label{fig:entanglement}
\end{figure}
Zucchetti S.p.a. è la prima software house in Italia per fatturato, opera nel settore dell'Information Technology, ed è stata fondata nel 1978 da Fabrizio Bernini a Lodi. 
L'azienda è specializzata nella realizzazione di software gestionali per pianificazione delle risorse d'impresa, soluzioni per il controllo degli accessi e sistemi di automazione industriale. 
Al giorno d'oggi Zucchetti oltre ad avere sedi in tutta Italia è presente anche in 50 paesi esteri, tra cui Cina, Germania, USA e Svizzera e conta più di 8.000 dipendenti e più di 1650 partner.


\section{Il progetto}
Il progetto si è svolto nella sede di padova di Zucchetti, in particolare nel reparto ...
Il progetto ha avuto come obiettivo la realizzazione di test derivati direttamente dal codice sorgente e dalla documentazione del progetto 
sfruttando le abilità dei sistemi di intelligenza artificiale ed in particolare dai Large Language Model\footcite{article:spooky}.
L'interesse inoltre era di verificare se i  modelli Open Source sono in grado di svolgere questo compito e in seguito definire la potenza necessaria dell'hardware (CPU, RAM, GPU) per prestazioni soddisfacenti.

\section{Strumenti utilizzati}
gli strumenti utilizzati per la realizzazione del progetto sono stati:
\begin{itemize}
    \item \textbf{Python}: linguaggio di programmazione utilizzato per la realizzazione dei modelli di Machine Learning;
    \item \textbf{Hugging Face}: libreria Python che fornisce modelli di Machine Learning pre-addestrati;
    \item \textbf{Docker}: strumento che permette di creare, testare e distribuire applicazioni in container;
    \item \textbf{Git}: sistema di controllo di versione distribuito;
    \item \textbf{GitHub}: servizio di hosting per progetti software che utilizzano Git;
    \item \textbf{Jupyter Notebook}: ambiente di sviluppo open-source per la creazione di documenti che contengono codice sorgente, equazioni, visualizzazioni e testo narrativo;
    \item \textbf{Google Colab}: servizio di Google che permette di eseguire codice Python in cloud;
    \item \textbf{PyCharm}: IDE per lo sviluppo in Python.
    \item \textbf{TensorFlow}: libreria open-source per il Machine Learning;
\end{itemize}

\section{Organizzazione del testo}
\begin{description}
    \item[{\hyperref[chap:processi-metodologie]{Il secondo capitolo}}] descrive ...
    
    \item[{\hyperref[chap:descrizione-stage]{Il terzo capitolo}}] approfondisce ...
    
    \item[{\hyperref[chap:analisi-requisiti]{Il quarto capitolo}}] approfondisce ...
    
    \item[{\hyperref[chap:progettazione-codifica]{Il quinto capitolo}}] approfondisce ...
    
    \item[{\hyperref[chap:verifica-validazione]{Il sesto capitolo}}] approfondisce ...
    
    \item[{\hyperref[chap:conclusioni]{Nel settimo capitolo}}] descrive ...
\end{description}

Riguardo la stesura del testo, relativamente al documento sono state adottate le seguenti convenzioni tipografiche:
\begin{itemize}
	\item gli acronimi, le abbreviazioni e i termini ambigui o di uso non comune menzionati vengono definiti nel glossario, situato alla fine del presente documento;
	\item per la prima occorrenza dei termini riportati nel glossario viene utilizzata la seguente nomenclatura: \textit{parola}\glox\gloxspacing;
	\item i termini in lingua straniera o facenti parti del gergo tecnico sono evidenziati con il carattere \textit{corsivo}.
\end{itemize}

\newpage