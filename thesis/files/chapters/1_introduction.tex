\chapter{Introduzione}
\label{chap:introduzione}



%Introduzione al contesto applicativo.

%Lorem Figure \ref{fig:entanglement}

%Esempio di utilizzo di un termine nel glossario \gls{api}.

%Esempio di citazione nel piè di pagina citazione\footcite{womak:lean-thinking}

%\lipsum[1-2]

\section{L'azienda}
\begin{figure}[h!]
    \centering
    \includegraphics[alt={Testo alternativo dell'immagine}, width=0.7\columnwidth]{img/logoZucchetti.jpeg}
    \caption{logo di Zucchetti}
    \label{fig:entanglement}
\end{figure}
Zucchetti S.p.a. è la prima software house in Italia per fatturato, opera nel settore dell'\textit{Information Technology}, ed è stata fondata nel 1978 da Fabrizio Bernini a Lodi. 
L'azienda è specializzata nella realizzazione di software gestionali per pianificazione delle risorse d'impresa, soluzioni per il controllo degli accessi e sistemi di automazione industriale. 
Al giorno d'oggi Zucchetti oltre ad avere sedi in tutta Italia è presente anche in 50 paesi esteri, tra cui Cina, Germania, USA e Svizzera e conta più di 8.000 dipendenti e più di 1650 \textit{partner}.

\section{Il progetto}
Lo \textit{stage} si svolgerà presso l'azienda Zucchetti, con sede a Padova. Il progetto di \textit{stage} prevede la ricerca e lo sviluppo di test automatici derivanti direttamente dal codice e dalla documentazione, sfruttando le abilità dei sistemi di \textit{intelligenza artificiale} ed in particolare dei \glspl{llmg}.
Lo \textit{stage} sarà diviso in due macroperiodi di quattro settimane ciascuno. Nel primo periodo dovrò analizzare attraverso lo studio di paper accademici e documentazioni le tecniche di \textit{testing} che sfruttano \glspl{llmg}. Successivamente dovrò implementare un \gls{prototipog} di generatore di test automatici in \textit{Python} che usufruisce di un modello basato su \textit{natural language processing}. Nella seconda parte del primo periodo dovrò decorare il codice attraverso commenti e generare \textit{test}. I risultati di questi verranno poi confrontati con quelli ottenuti dalla prima parte di periodo. Nel secondo periodo invece farò \gls{fine-tuning} dei modelli \glspl{llm} con il metodo \gls{lora}, per poi confrontare i risultati ottenuti con quelli nel primo periodo.
Viene richiesto anche lo studio di tecniche di \textit{quantizzazione} per ridurre la dimensione dei modelli \glspl{llm} e la loro complessità computazionale.
\section{Organizzazione del testo}
\begin{description}
    \item[{\hyperref[chap:processi-metodologie]{Il secondo capitolo}}] descrive i processi e le metodologie utilizzate durante lo \textit{stage}. In particolare si approfondiranno i processi di sviluppo \textit{software} e gli strumenti utilizzati per fare ciò.
    
    \item[{\hyperref[chap:descrizione-stage-1]{Il terzo capitolo}}] si propone di delineare il dominio applicativo del progetto, mediante un'analisi dettagliata del tema accompagnata da esempi pratici di utilizzo. 
    Inoltre, si provvederà a fornire una descrizione esaustiva del funzionamento di \textit{Assured LLMS}. 
    In questa sezione, sarà altresì redatta una lista esaustiva di rischi, requisiti e obiettivi del progetto. 
    Successivamente, si procederà con la descrizione del prodotto sviluppato, che includerà \textit{script} e \textit{benchmarks}. 

    \item[{\hyperref[chap:descrizione-stage-2]{Il quarto capitolo}}] descrive il processo di applicazione di \gls{lora} e le possibili ottimizzazioni, andando ad approfondire le tecniche utilizzate durante il processo.

    \item[{\hyperref[chap:conclusioni]{Il quinto capitolo}}] concluderà il documento, presentando una valutazione retrospettiva personale dell'esperienza di \textit{stage} e delle conoscenze acquisite.
    
\end{description}

In seguito si possono trovare le convenzioni tipografiche utilizzate per la stesura del documento:
\begin{itemize}
	\item gli acronimi, le abbreviazioni e i termini ambigui o di uso non comune menzionati vengono definiti nel glossario, situato alla fine del presente documento;
	\item per la prima occorrenza dei termini riportati nel glossario viene utilizzata la seguente nomenclatura: \textit{parola}\glox\gloxspacing;
	\item i termini in lingua straniera o facenti parti del gergo tecnico sono evidenziati con il carattere \textit{corsivo}.
\end{itemize}

\newpage