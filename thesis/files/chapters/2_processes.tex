\chapter{Processi e metodologie}
\label{chap:processi-metodologie}

\section{Processo di sviluppo del prodotto}
Il processo di sviluppo del prodotto è iniziato con la ricerca di articoli accademici inerenti agli argomenti sui quali si sarebbe dovuto basare il prodotto. Successivamente si è proceduto con la lettura e la comprensione degli stessi, integrando le conoscenze attraverso libri di \gls{mclearning} e \gls{deeplearning}.
Durante le ore lavorative vi è stata inoltre la possibilità di accrescere le mie conoscenze teoriche e pratiche grazie all'aiuto dei colleghi i quali, sin da subito, hanno mostrato interesse nell'argomento.
Nell'ufficio di ricerca e sviluppo di Zucchetti, ove ho svolto lo stage, si lavora in un ambiente rilassato ma allo stesso tempo incentrato a portare valore al prodotto, questo mi ha spinto a lavorare in modo creativo e costantemente alla ricerca di accrescere le mie conoscenze.
Oltre a ciò, vi è stata la possibilità di constatare le consocenze apprese attraverso esposizioni durante le riuniuoni interne all'azienda. Questi \textit{meeting} mi hanno quindi permesso di verificare la comprensione dell'argomento e di ricevere feedback da parte dei colleghi.

Durante la fase di sviluppo invece il lavoro è stato svolto in autonomia, con la possibilità di confrontarmi con i colleghi in caso di dubbi o problemi.
Mi sono inoltre confrontato con gli altri stagisti che lavoravano su progetti simili, per fare \textit{brain-storming} e per discutere delle soluzioni da noi adottate alla ricerca di idee e soluzioni a difficoltà comuni.
Infine, il prodotto è stato testato autonomamente e successivamente con i colleghi per valutare la qualità del prodotto e ricevere feedback per eventuali miglioramenti.


% come si è lavorato per sviluppare il prodotto ???
\section{Strumenti utilizzati}
gli strumenti utilizzati per la realizzazione del progetto sono stati:
\begin{itemize}
    %\item \textbf{Docker}: strumento che permette di creare, testare e distribuire applicazioni in container;
    \item \textbf{Git}: sistema di controllo di versione distribuito;
    \item \textbf{GitHub}: servizio di hosting per progetti software che utilizzano Git;
    \item \textbf{Google Colab Pro}: servizio di Google che permette di eseguire codice Python in cloud;
    \item \textbf{Hugging Face}: libreria Python che fornisce modelli di Machine Learning pre-addestrati;
    \item \textbf{LM Studio}: software che permette di scaricare ed utilizzare localmente alcuni dei modelli di Hugging Face;
    \item \textbf{PyCharm}: \gls{ide}\glox per lo sviluppo in Python;
    \item \textbf{Python}: linguaggio di programmazione ad alto livello, interpretato, interattivo, orientato agli oggetti, adatto per lo sviluppo di applicazioni legate al machine learning;
    %\item \textbf{TensorFlow}: libreria open-source per il Machine Learning;
\end{itemize}

\newpage